% Rxtraido de https://github.com/caio1982/Monografia-Latex
% Portanto, creditos ao Caio Begotti e todos antes dele  

% Numeração de acordo com UFPR
\documentclass[pnumromarab,normaltoc]{abnt}

% Para inserir palavras com alfabeto grego no texto
\usepackage[polutonikogreek,brazil]{babel}

% Pra suportar acentuação direito

\usepackage[latin1]{inputenc}
\usepackage{abnt-alf}

% Para inserção de imagens
\usepackage{graphicx}

% Para links web serem clicáveis
\usepackage{url}
\usepackage{remreset}
\usepackage[normalem]{ulem}

% Força uso da font Computer Modern
\usepackage[T1]{fontenc}
\usepackage{textcomp}
\usepackage{lmodern} % http://www.tug.dk/FontCatalogue/lmodern/

% Minted para coloração de sintaxe dos códigos de programação (syntax highlighting)
\usepackage{minted}

% Estilo preferido do Minted pra Python
\usemintedstyle{bw}
\renewcommand{\theFancyVerbLine}{\sffamily \textcolor[rgb]{0.5,0.5,0.5}{\scriptsize \oldstylenums{\arabic{FancyVerbLine}}}}

\makeatletter
\@removefromreset{footnote}{chapter}
\makeatother

% Para que ele entenda o @
\makeatletter

% Capa da monografia
\renewcommand{\capa} {
\begin{titlepage}
	\espaco{1.1}
	
	\begin{center}
		\large\ABNTchapterfont\ABNTautordata
	\end{center}
	
	\vspace{7.5cm}
	
	\begin{center}
		\large\ABNTchapterfont\ABNTtitulodata\par
	\end{center}
	
	\vfill
	
	\begin{center}
		\textbf{\ABNTlocaldata}\par
		\textbf{\ABNTdatadata}
	\end{center}
\end{titlepage}
}

% Folha de rosto
\newcommand{\esporient}[2] {
	\leftskip 0em
	\@tempdima 5.5em
	\advance\leftskip \@tempdima \null\nobreak\hskip -\leftskip
	{#1#2\hfil}
}

\newcommand{\espcoorient}[2] {
	\leftskip 0em
	\@tempdima 7em
	\advance\leftskip \@tempdima \null\nobreak\hskip -\leftskip
	{#1#2\hfil}
}

\renewcommand{\folhaderosto} {
\begin{titlepage}
	\espaco{1.1}
	
	\begin{center}
		\large\ABNTchapterfont\ABNTautordata
	\end{center}
	
	\vspace{7.5cm}
	
	\begin{center}
		\large\ABNTchapterfont\ABNTtitulodata\par
	\end{center}
	
	\vspace{2cm}
	
	\hspace{.35\textwidth}
	\begin{minipage}{.5\textwidth}
		\begin{espacosimples}
			\ABNTcomentariodata\par
		\end{espacosimples}
	\end{minipage}
	
	\hspace{.35\textwidth}
	\begin{minipage}{.5\textwidth}
		\begin{espacosimples}
			\esporient{\numberline {Orientador:}}{\ignorespaces\ABNTorientadordata}
		\end{espacosimples}
	\end{minipage}
	
	\ABNTifnotempty{\ABNTcoorientadordata}{
		\hspace{.35\textwidth}
		\begin{minipage}{.5\textwidth}
			\begin{espacosimples}
				\espcoorient{\numberline {Co-Orientador:}}{\ignorespaces\ABNTcoorientadordata}
			\end{espacosimples}
		\end{minipage}}
	
	\vfill
	
	\begin{center}
		\textbf{\ABNTlocaldata}\par
		\textbf{\ABNTdatadata}
	\end{center}

\end{titlepage}
}

% Altera o tamanho das fontes dos capítulos e dos apêndices
\renewcommand{\ABNTchapterfont}{\bfseries}
\renewcommand{\ABNTchaptersize}{\Large}
\renewcommand{\ABNTanapsize}{\Large}

% Altera o espaçamento entre dots
\renewcommand\@dotsep{2}

% Altera forma de montagem do TOC
\renewcommand\l@chapter[2]{
  \ifnum \c@tocdepth >\m@ne
    \addpenalty{-\@highpenalty}%
    \vskip 1.0em \@plus\p@
    \setlength\@tempdima{1.5em}%
    \begingroup
      \ifthenelse{\boolean{ABNTpagenumstyle}}
        {\renewcommand{\@pnumwidth}{3.5em}}
        {}
      \parindent \z@ \rightskip \@pnumwidth
      \parfillskip -\@pnumwidth
      \leavevmode \normalsize\ABNTtocchapterfont
      \advance\leftskip\@tempdima
      \hskip -\leftskip
      #1\nobreak\dotfill \nobreak%
      \ifthenelse{\boolean{ABNTpagenumstyle}}
         {%
          \hb@xt@\@pnumwidth{\hss 
            \ifthenelse{\not\equal{#2}{}}{{\normalfont p.\thinspace#2}}{}}\par
         }
         {%
          \hb@xt@\@pnumwidth{\hss #2}\par
         }
      \penalty\@highpenalty
    \endgroup
  \fi
}

\renewcommand*\l@section{\@dottedtocline{1}{0em}{2.3em}}
\renewcommand*\l@subsection{\@dottedtocline{2}{0em}{3.2em}}
\renewcommand*\l@subsubsection{\@dottedtocline{3}{0em}{4.1em}}

% Cria um comando auxiliar para montagem da lista de figuras
\newcommand{\figfillnum}[1]{%
  {\hspace{1em}\normalfont\dotfill}\nobreak
  \hb@xt@\@pnumwidth{\hfil\normalfont #1}{}\par}

% Cria um comando auxiliar para montagem da lista de tabelas
\newcommand{\tabfillnum}[1]{%
	{\hspace{1em}\normalfont\dotfill}\nobreak
	\hb@xt@\@pnumwidth{\hfil\normalfont #1}{}\par}

% Altera a forma de montagem da lista de figuras
\renewcommand*{\l@figure}[2]{
	\leftskip 3.1em
	\rightskip 1.6em
	\parfillskip -\rightskip
	\parindent 0em
	\@tempdima 2.0em
	\advance\leftskip \@tempdima \null\nobreak\hskip -\leftskip
	{Figura \normalfont #1}\nobreak \figfillnum{#2}}

% Altera a forma de montagem de lista de tabelas
\renewcommand*{\l@table}[2]{
	\leftskip 3.4em
	\rightskip 1.6em
	\parfillskip -\rightskip
	\parindent 0em
	\@tempdima 2.0em
	\advance\leftskip \@tempdima \null\nobreak\hskip -\leftskip
	{Tabela \normalfont #1}\nobreak \tabfillnum{#2}}

% Define os comandos que montam a lista de símbolos
\newcommand{\listadesimbolos}{\pretextualchapter{Lista de Símbolos}\@starttoc{lsb}}
\newcommand{\simbolo}[2]{{\addcontentsline{lsb}{simbolo}{\numberline{#1}{#2}}}#1}
\newcommand{\l@simbolo}[2]{
	\vspace{-0.75cm}
	\leftskip 0em
	\parindent 0em
	\@tempdima 5em
	\advance\leftskip \@tempdima \null\nobreak\hskip -\leftskip
	{\normalfont #1}\hfil\nobreak\par}

% Define o comando que monta a lista de siglas
\newcommand{\listadesiglas}{\pretextualchapter{Lista de Siglas}\@starttoc{lsg}}
\newcommand{\sigla}[2]{{\addcontentsline{lsg}{sigla}{\numberline{#1}{#2}}}#1}
\newcommand{\l@sigla}[2]{
	\vspace{-0.75cm}
	\leftskip 0em
	\parindent 0em
	\@tempdima 5em
	\advance\leftskip \@tempdima \null\nobreak\hskip -\leftskip
	{\normalfont #1}\hfil\nobreak\par}

% Define o tipo de numeração das páginas
\renewcommand{\chaptertitlepagestyle}{plain}

% Altera a posição da numeração de páginas dos elementos pré-textuais
\renewcommand\pretextualchapter{
	\if@openright\cleardoublepage\else\clearpage\fi
	\pagestyle{\chaptertitlepagestyle}
	\global\@topnum\z@
	\@afterindentfalse
	\@schapter}

% Altera a posição da numeração de páginas dos elementos textuais
\renewcommand{\ABNTchaptermark}[1]{
	\ifthenelse{\boolean{ABNTNextOutOfTOC}}
		{\markboth{\ABNTnextmark}{\ABNTnextmark}}
		{\chaptermark{#1}
		\pagestyle{\chaptertitlepagestyle}}}

% Redefine o tipo de numeração das páginas
\renewcommand{\ABNTBeginOfTextualPart}{
	\renewcommand{\chaptertitlepagestyle}{plainheader}
	\renewcommand{\thepage}{\arabic{page}}
	\setcounter{page}{1}}

\makeatother

% Altera o tamanho do parágrafo
\setlength{\parindent}{1.5cm}

